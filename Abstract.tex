\documentclass{ansnse}
\usepackage{authblk}
\usepackage{amsmath}
\usepackage{siunitx}
\usepackage{isotope}
\usepackage[margin=1in]{geometry}
\usepackage{setspace}
\usepackage{verbatim}
\usepackage{graphicx}
\usepackage{todonotes}
\usepackage{subfig}
\usepackage{booktabs}

\usepackage{hyperref}
\hypersetup{colorlinks=true,
citecolor=black,
linkcolor=black}

\sisetup{list-final-separator={, and },
separate-uncertainty=true,
range-phrase={--},
multi-part-units=single,
list-units=single,
range-units=single}

\author{Jeremy Lloyd Conlin\footnote{\texttt{jlconlin@lanl.gov}} }  % PNAR, MC
\author{Stephen J. Tobin}   % Overall effort
\author{Adrienne M. LaFleur}    % SINRD
\author{Jianwei Hu}         % CIPN
\author{TaeHoon Lee}        % DDA
\author{Nathan P. Sandoval} % DN
\author{Melissa A. Schear}  % DDSI

\affil{\normalsize\emph{Los Alamos National Laboratory,
Los Alamos, NM 87544}}

\title{On Using Code Emulators and Monte Carlo Estimation to Predict Assembly Attributes of Spent Fuel Assemblies for Safeguards Applications}

\date{}

\DeclareMathAlphabet{\mathpzc}{OT1}{pzc}{m}{it}

\newcommand{\PuEff}{\ensuremath{\isotope[239]{Pu}_{\text{e}}}}
\newcommand{\Pub}{\ensuremath{\PuEff'}}
\newcommand{\dd}{\ensuremath{\mathop{}\!\mathrm{d}}}
\newcommand{\weight}{\ensuremath{\mathpzc{w}}}

\begin{document}
\ansabstract{The quantification of the plutonium mass in spent nuclear fuel assemblies is an important measurement for nuclear safeguards practitioners.  A program is well underway to develop nondestructive assay instruments which, when combined, will be able to quantify the plutonium content of a spent nuclear fuel assembly.  Each instrument will quantify a specific attribute of the spent fuel assembly e.g., the fissile content.  In this paper, we present a Monte Carlo-based method of estimating the mean and distribution of some assembly attributes.  An MCNPX model of each instrument has been created and the response of the instrument was simulated for a range of spent fuel assemblies with discrete parameters (e.g., burnup, initial enrichment, and cooling time).  The Monte Carlo-based method interpolates between the modeled results for an instrument to emulate a response for parameters not explicitly modeled.  We demonstrate the usefulness of this technique in applying the technique to six different instruments under investigation. The results show that this Monte Carlo-based method can be used to estimate the assembly attributes of a spent fuel assembly based upon the measured response from the instrument.
}
keywords: Monte Carlo, emulation, safeguards.
\end{document}

